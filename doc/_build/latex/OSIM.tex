%% Generated by Sphinx.
\def\sphinxdocclass{report}
\documentclass[letterpaper,12pt,english]{sphinxmanual}
\ifdefined\pdfpxdimen
   \let\sphinxpxdimen\pdfpxdimen\else\newdimen\sphinxpxdimen
\fi \sphinxpxdimen=.75bp\relax

\usepackage[utf8]{inputenc}
\ifdefined\DeclareUnicodeCharacter
 \ifdefined\DeclareUnicodeCharacterAsOptional\else
  \DeclareUnicodeCharacter{00A0}{\nobreakspace}
\fi\fi
\usepackage{cmap}
\usepackage[T1]{fontenc}
\usepackage{amsmath,amssymb,amstext}
\usepackage{babel}
\usepackage{times}
\usepackage[Bjarne]{fncychap}
\usepackage{longtable}
\usepackage{sphinx}

\usepackage{geometry}
\usepackage{multirow}
\usepackage{eqparbox}

% Include hyperref last.
\usepackage{hyperref}
% Fix anchor placement for figures with captions.
\usepackage{hypcap}% it must be loaded after hyperref.
% Set up styles of URL: it should be placed after hyperref.
\urlstyle{same}
\addto\captionsenglish{\renewcommand{\contentsname}{Contents:}}

\addto\captionsenglish{\renewcommand{\figurename}{Fig.}}
\addto\captionsenglish{\renewcommand{\tablename}{Table}}
\addto\captionsenglish{\renewcommand{\literalblockname}{Listing}}

\addto\extrasenglish{\def\pageautorefname{page}}

\setcounter{tocdepth}{0}



\title{OSIM Documentation}
\date{May 25, 2017}
\release{0.1}
\author{Tim Maiwald}
\newcommand{\sphinxlogo}{}
\renewcommand{\releasename}{Release}
\makeindex

\begin{document}

\maketitle
\sphinxtableofcontents
\phantomsection\label{\detokenize{index::doc}}


OSIM is a analog cicruit simulation tool with the main target to explore new circuit optimization algorithms and methods. It can be separated into three layer:


\chapter{Modeling}
\label{\detokenize{Modeling:welcome-to-osim-s-documentation}}\label{\detokenize{Modeling::doc}}\label{\detokenize{Modeling:modeling}}
The Modeling-Layer is the fundamental part of the mathematical description of an analog curcuit. Therefor exist two again


\section{Circuit System Equations}
\label{\detokenize{Modeling/CircuitSystemEquations:module-Modeling.CircuitSystemEquations}}\label{\detokenize{Modeling/CircuitSystemEquations::doc}}\label{\detokenize{Modeling/CircuitSystemEquations:circuit-system-equations}}\index{Modeling.CircuitSystemEquations (module)}\index{CircuitSystemEquations (class in Modeling.CircuitSystemEquations)}

\begin{fulllineitems}
\phantomsection\label{\detokenize{Modeling/CircuitSystemEquations:Modeling.CircuitSystemEquations.CircuitSystemEquations}}\pysiglinewithargsret{\sphinxstrong{class }\sphinxcode{Modeling.CircuitSystemEquations.}\sphinxbfcode{CircuitSystemEquations}}{\emph{components}}{}
Bases: \sphinxcode{object}

Class that contains the main data structures for a curcuit
\index{ATYPE\_AC (Modeling.CircuitSystemEquations.CircuitSystemEquations attribute)}

\begin{fulllineitems}
\phantomsection\label{\detokenize{Modeling/CircuitSystemEquations:Modeling.CircuitSystemEquations.CircuitSystemEquations.ATYPE_AC}}\pysigline{\sphinxbfcode{ATYPE\_AC}\sphinxstrong{ = 3}}
\end{fulllineitems}

\index{ATYPE\_DC (Modeling.CircuitSystemEquations.CircuitSystemEquations attribute)}

\begin{fulllineitems}
\phantomsection\label{\detokenize{Modeling/CircuitSystemEquations:Modeling.CircuitSystemEquations.CircuitSystemEquations.ATYPE_DC}}\pysigline{\sphinxbfcode{ATYPE\_DC}\sphinxstrong{ = 2}}
\end{fulllineitems}

\index{ATYPE\_EST\_DC (Modeling.CircuitSystemEquations.CircuitSystemEquations attribute)}

\begin{fulllineitems}
\phantomsection\label{\detokenize{Modeling/CircuitSystemEquations:Modeling.CircuitSystemEquations.CircuitSystemEquations.ATYPE_EST_DC}}\pysigline{\sphinxbfcode{ATYPE\_EST\_DC}\sphinxstrong{ = 5}}
\end{fulllineitems}

\index{ATYPE\_NONE (Modeling.CircuitSystemEquations.CircuitSystemEquations attribute)}

\begin{fulllineitems}
\phantomsection\label{\detokenize{Modeling/CircuitSystemEquations:Modeling.CircuitSystemEquations.CircuitSystemEquations.ATYPE_NONE}}\pysigline{\sphinxbfcode{ATYPE\_NONE}\sphinxstrong{ = 1}}
\end{fulllineitems}

\index{ATYPE\_TRAN (Modeling.CircuitSystemEquations.CircuitSystemEquations attribute)}

\begin{fulllineitems}
\phantomsection\label{\detokenize{Modeling/CircuitSystemEquations:Modeling.CircuitSystemEquations.CircuitSystemEquations.ATYPE_TRAN}}\pysigline{\sphinxbfcode{ATYPE\_TRAN}\sphinxstrong{ = 4}}
\end{fulllineitems}

\index{getCompByName() (Modeling.CircuitSystemEquations.CircuitSystemEquations method)}

\begin{fulllineitems}
\phantomsection\label{\detokenize{Modeling/CircuitSystemEquations:Modeling.CircuitSystemEquations.CircuitSystemEquations.getCompByName}}\pysiglinewithargsret{\sphinxbfcode{getCompByName}}{\emph{name}}{}
\end{fulllineitems}

\index{getPreviousSolutionAt() (Modeling.CircuitSystemEquations.CircuitSystemEquations method)}

\begin{fulllineitems}
\phantomsection\label{\detokenize{Modeling/CircuitSystemEquations:Modeling.CircuitSystemEquations.CircuitSystemEquations.getPreviousSolutionAt}}\pysiglinewithargsret{\sphinxbfcode{getPreviousSolutionAt}}{\emph{componentName}}{}
\end{fulllineitems}

\index{getSolutionAt() (Modeling.CircuitSystemEquations.CircuitSystemEquations method)}

\begin{fulllineitems}
\phantomsection\label{\detokenize{Modeling/CircuitSystemEquations:Modeling.CircuitSystemEquations.CircuitSystemEquations.getSolutionAt}}\pysiglinewithargsret{\sphinxbfcode{getSolutionAt}}{\emph{componentName}}{}
method to get a value out of the solution vector x
\begin{quote}\begin{description}
\item[{Parameters}] \leavevmode
\sphinxstyleliteralstrong{componentName} (\sphinxstyleliteralemphasis{string}) -- name of a component or a node in the netlist

\item[{Returns}] \leavevmode
solution

\item[{Return type}] \leavevmode
complex

\end{description}\end{quote}

\end{fulllineitems}

\index{isVoltage() (Modeling.CircuitSystemEquations.CircuitSystemEquations method)}

\begin{fulllineitems}
\phantomsection\label{\detokenize{Modeling/CircuitSystemEquations:Modeling.CircuitSystemEquations.CircuitSystemEquations.isVoltage}}\pysiglinewithargsret{\sphinxbfcode{isVoltage}}{\emph{name}}{}
\end{fulllineitems}

\index{printComponents() (Modeling.CircuitSystemEquations.CircuitSystemEquations method)}

\begin{fulllineitems}
\phantomsection\label{\detokenize{Modeling/CircuitSystemEquations:Modeling.CircuitSystemEquations.CircuitSystemEquations.printComponents}}\pysiglinewithargsret{\sphinxbfcode{printComponents}}{}{}
\end{fulllineitems}

\index{reset() (Modeling.CircuitSystemEquations.CircuitSystemEquations method)}

\begin{fulllineitems}
\phantomsection\label{\detokenize{Modeling/CircuitSystemEquations:Modeling.CircuitSystemEquations.CircuitSystemEquations.reset}}\pysiglinewithargsret{\sphinxbfcode{reset}}{}{}
method to reset the equations

\end{fulllineitems}

\index{setParameterForCompsList() (Modeling.CircuitSystemEquations.CircuitSystemEquations method)}

\begin{fulllineitems}
\phantomsection\label{\detokenize{Modeling/CircuitSystemEquations:Modeling.CircuitSystemEquations.CircuitSystemEquations.setParameterForCompsList}}\pysiglinewithargsret{\sphinxbfcode{setParameterForCompsList}}{\emph{setables}}{}~\begin{quote}\begin{description}
\item[{Parameters}] \leavevmode
\sphinxstyleliteralstrong{setables} -- list({[}compname,paramname,paramval{]},{[}...{]})

\end{description}\end{quote}

\end{fulllineitems}

\index{setParamterForComp() (Modeling.CircuitSystemEquations.CircuitSystemEquations method)}

\begin{fulllineitems}
\phantomsection\label{\detokenize{Modeling/CircuitSystemEquations:Modeling.CircuitSystemEquations.CircuitSystemEquations.setParamterForComp}}\pysiglinewithargsret{\sphinxbfcode{setParamterForComp}}{\emph{compname}, \emph{paramname}, \emph{paramval}}{}
\end{fulllineitems}

\index{setSolutionAt() (Modeling.CircuitSystemEquations.CircuitSystemEquations method)}

\begin{fulllineitems}
\phantomsection\label{\detokenize{Modeling/CircuitSystemEquations:Modeling.CircuitSystemEquations.CircuitSystemEquations.setSolutionAt}}\pysiglinewithargsret{\sphinxbfcode{setSolutionAt}}{\emph{name}, \emph{val}}{}
\end{fulllineitems}


\end{fulllineitems}



\section{Components}
\label{\detokenize{Modeling/Components::doc}}\label{\detokenize{Modeling/Components:components}}

\subsection{OSIM.Modeling.AbstractComponents.Component module}
\label{\detokenize{Modeling/Components:osim-modeling-abstractcomponents-component-module}}\label{\detokenize{Modeling/Components:module-Modeling.AbstractComponents.Component}}\index{Modeling.AbstractComponents.Component (module)}\index{Component (class in Modeling.AbstractComponents.Component)}

\begin{fulllineitems}
\phantomsection\label{\detokenize{Modeling/Components:Modeling.AbstractComponents.Component.Component}}\pysiglinewithargsret{\sphinxstrong{class }\sphinxcode{Modeling.AbstractComponents.Component.}\sphinxbfcode{Component}}{\emph{nodes}, \emph{name}, \emph{value}, \emph{superComponent}, \emph{**kwargs}}{}
Bases: \sphinxcode{object}

Abstract description of a component in a circuit
\index{GMIN (Modeling.AbstractComponents.Component.Component attribute)}

\begin{fulllineitems}
\phantomsection\label{\detokenize{Modeling/Components:Modeling.AbstractComponents.Component.Component.GMIN}}\pysigline{\sphinxbfcode{GMIN}\sphinxstrong{ = 1e-09}}
\end{fulllineitems}

\index{Udiff() (Modeling.AbstractComponents.Component.Component method)}

\begin{fulllineitems}
\phantomsection\label{\detokenize{Modeling/Components:Modeling.AbstractComponents.Component.Component.Udiff}}\pysiglinewithargsret{\sphinxbfcode{Udiff}}{\emph{twonodes}}{}
\end{fulllineitems}

\index{assignToSystem() (Modeling.AbstractComponents.Component.Component method)}

\begin{fulllineitems}
\phantomsection\label{\detokenize{Modeling/Components:Modeling.AbstractComponents.Component.Component.assignToSystem}}\pysiglinewithargsret{\sphinxbfcode{assignToSystem}}{\emph{sys}}{}
\end{fulllineitems}

\index{containsNonlinearity() (Modeling.AbstractComponents.Component.Component method)}

\begin{fulllineitems}
\phantomsection\label{\detokenize{Modeling/Components:Modeling.AbstractComponents.Component.Component.containsNonlinearity}}\pysiglinewithargsret{\sphinxbfcode{containsNonlinearity}}{}{}
\end{fulllineitems}

\index{doStep() (Modeling.AbstractComponents.Component.Component method)}

\begin{fulllineitems}
\phantomsection\label{\detokenize{Modeling/Components:Modeling.AbstractComponents.Component.Component.doStep}}\pysiglinewithargsret{\sphinxbfcode{doStep}}{\emph{freq\_or\_tau}}{}
abstract function to use in a iteration of simulations
\begin{quote}\begin{description}
\item[{Parameters}] \leavevmode
\sphinxstyleliteralstrong{freq\_or\_tau} (\sphinxstyleliteralemphasis{float}) -- frequency or timestep value depending on simulation type

\end{description}\end{quote}

\end{fulllineitems}

\index{getAdmittance() (Modeling.AbstractComponents.Component.Component method)}

\begin{fulllineitems}
\phantomsection\label{\detokenize{Modeling/Components:Modeling.AbstractComponents.Component.Component.getAdmittance}}\pysiglinewithargsret{\sphinxbfcode{getAdmittance}}{\emph{nodesFromTo}, \emph{freq\_or\_tstep}}{}
\end{fulllineitems}

\index{getMyParameterFromDictionary() (Modeling.AbstractComponents.Component.Component method)}

\begin{fulllineitems}
\phantomsection\label{\detokenize{Modeling/Components:Modeling.AbstractComponents.Component.Component.getMyParameterFromDictionary}}\pysiglinewithargsret{\sphinxbfcode{getMyParameterFromDictionary}}{\emph{param}, \emph{paramDict}, \emph{default}}{}
\end{fulllineitems}

\index{getNodes() (Modeling.AbstractComponents.Component.Component method)}

\begin{fulllineitems}
\phantomsection\label{\detokenize{Modeling/Components:Modeling.AbstractComponents.Component.Component.getNodes}}\pysiglinewithargsret{\sphinxbfcode{getNodes}}{}{}
\end{fulllineitems}

\index{getValue() (Modeling.AbstractComponents.Component.Component method)}

\begin{fulllineitems}
\phantomsection\label{\detokenize{Modeling/Components:Modeling.AbstractComponents.Component.Component.getValue}}\pysiglinewithargsret{\sphinxbfcode{getValue}}{}{}
\end{fulllineitems}

\index{initialSignIntoSysEquations() (Modeling.AbstractComponents.Component.Component method)}

\begin{fulllineitems}
\phantomsection\label{\detokenize{Modeling/Components:Modeling.AbstractComponents.Component.Component.initialSignIntoSysEquations}}\pysiglinewithargsret{\sphinxbfcode{initialSignIntoSysEquations}}{}{}
\end{fulllineitems}

\index{insertAdmittanceintoSystem() (Modeling.AbstractComponents.Component.Component method)}

\begin{fulllineitems}
\phantomsection\label{\detokenize{Modeling/Components:Modeling.AbstractComponents.Component.Component.insertAdmittanceintoSystem}}\pysiglinewithargsret{\sphinxbfcode{insertAdmittanceintoSystem}}{\emph{freq}}{}
\end{fulllineitems}

\index{myBranchCurrent() (Modeling.AbstractComponents.Component.Component method)}

\begin{fulllineitems}
\phantomsection\label{\detokenize{Modeling/Components:Modeling.AbstractComponents.Component.Component.myBranchCurrent}}\pysiglinewithargsret{\sphinxbfcode{myBranchCurrent}}{}{}
\end{fulllineitems}

\index{parseArgs() (Modeling.AbstractComponents.Component.Component method)}

\begin{fulllineitems}
\phantomsection\label{\detokenize{Modeling/Components:Modeling.AbstractComponents.Component.Component.parseArgs}}\pysiglinewithargsret{\sphinxbfcode{parseArgs}}{\emph{**kwargs}}{}
\end{fulllineitems}

\index{performCalculations() (Modeling.AbstractComponents.Component.Component method)}

\begin{fulllineitems}
\phantomsection\label{\detokenize{Modeling/Components:Modeling.AbstractComponents.Component.Component.performCalculations}}\pysiglinewithargsret{\sphinxbfcode{performCalculations}}{}{}
\end{fulllineitems}

\index{printMyOPValues() (Modeling.AbstractComponents.Component.Component method)}

\begin{fulllineitems}
\phantomsection\label{\detokenize{Modeling/Components:Modeling.AbstractComponents.Component.Component.printMyOPValues}}\pysiglinewithargsret{\sphinxbfcode{printMyOPValues}}{}{}
\end{fulllineitems}

\index{putA() (Modeling.AbstractComponents.Component.Component method)}

\begin{fulllineitems}
\phantomsection\label{\detokenize{Modeling/Components:Modeling.AbstractComponents.Component.Component.putA}}\pysiglinewithargsret{\sphinxbfcode{putA}}{\emph{A}, \emph{node}, \emph{myIdx}, \emph{nm}, \emph{mn}}{}
\end{fulllineitems}

\index{putJ() (Modeling.AbstractComponents.Component.Component method)}

\begin{fulllineitems}
\phantomsection\label{\detokenize{Modeling/Components:Modeling.AbstractComponents.Component.Component.putJ}}\pysiglinewithargsret{\sphinxbfcode{putJ}}{\emph{J}, \emph{m}, \emph{node}, \emph{mn}}{}
\end{fulllineitems}

\index{readParamsAndVariables() (Modeling.AbstractComponents.Component.Component method)}

\begin{fulllineitems}
\phantomsection\label{\detokenize{Modeling/Components:Modeling.AbstractComponents.Component.Component.readParamsAndVariables}}\pysiglinewithargsret{\sphinxbfcode{readParamsAndVariables}}{\emph{filename}}{}
ACHTUNG: BEIM AUSDRUECKE MIT VARIABLEN MUESSEN FLOATS ENTHALTEN UND KEINE INTEGER; WEIL SONST GEFAHR; DASS NICHT GERUNDET WIRD!!

\end{fulllineitems}

\index{registerBranches() (Modeling.AbstractComponents.Component.Component method)}

\begin{fulllineitems}
\phantomsection\label{\detokenize{Modeling/Components:Modeling.AbstractComponents.Component.Component.registerBranches}}\pysiglinewithargsret{\sphinxbfcode{registerBranches}}{\emph{compDict}, \emph{sysIdx}}{}
\end{fulllineitems}

\index{registerNodes() (Modeling.AbstractComponents.Component.Component method)}

\begin{fulllineitems}
\phantomsection\label{\detokenize{Modeling/Components:Modeling.AbstractComponents.Component.Component.registerNodes}}\pysiglinewithargsret{\sphinxbfcode{registerNodes}}{\emph{compDict}, \emph{sysIdx}}{}
\end{fulllineitems}

\index{reloadParams() (Modeling.AbstractComponents.Component.Component method)}

\begin{fulllineitems}
\phantomsection\label{\detokenize{Modeling/Components:Modeling.AbstractComponents.Component.Component.reloadParams}}\pysiglinewithargsret{\sphinxbfcode{reloadParams}}{}{}
\end{fulllineitems}

\index{setNewParamsAndVariablesDicts() (Modeling.AbstractComponents.Component.Component method)}

\begin{fulllineitems}
\phantomsection\label{\detokenize{Modeling/Components:Modeling.AbstractComponents.Component.Component.setNewParamsAndVariablesDicts}}\pysiglinewithargsret{\sphinxbfcode{setNewParamsAndVariablesDicts}}{\emph{paramDict}, \emph{variableDict}}{}
\end{fulllineitems}

\index{setOPValues() (Modeling.AbstractComponents.Component.Component method)}

\begin{fulllineitems}
\phantomsection\label{\detokenize{Modeling/Components:Modeling.AbstractComponents.Component.Component.setOPValues}}\pysiglinewithargsret{\sphinxbfcode{setOPValues}}{}{}
\end{fulllineitems}

\index{setParameterOrVariableValue() (Modeling.AbstractComponents.Component.Component method)}

\begin{fulllineitems}
\phantomsection\label{\detokenize{Modeling/Components:Modeling.AbstractComponents.Component.Component.setParameterOrVariableValue}}\pysiglinewithargsret{\sphinxbfcode{setParameterOrVariableValue}}{\emph{name}, \emph{value}}{}
\end{fulllineitems}

\index{setValue() (Modeling.AbstractComponents.Component.Component method)}

\begin{fulllineitems}
\phantomsection\label{\detokenize{Modeling/Components:Modeling.AbstractComponents.Component.Component.setValue}}\pysiglinewithargsret{\sphinxbfcode{setValue}}{\emph{value}}{}
\end{fulllineitems}

\index{signIntoSysDictionary() (Modeling.AbstractComponents.Component.Component method)}

\begin{fulllineitems}
\phantomsection\label{\detokenize{Modeling/Components:Modeling.AbstractComponents.Component.Component.signIntoSysDictionary}}\pysiglinewithargsret{\sphinxbfcode{signIntoSysDictionary}}{\emph{compDict}, \emph{sysIdx}}{}
\end{fulllineitems}

\index{updateParamsAndVariablesDicts() (Modeling.AbstractComponents.Component.Component method)}

\begin{fulllineitems}
\phantomsection\label{\detokenize{Modeling/Components:Modeling.AbstractComponents.Component.Component.updateParamsAndVariablesDicts}}\pysiglinewithargsret{\sphinxbfcode{updateParamsAndVariablesDicts}}{\emph{paramDict}, \emph{variableDict}}{}
\end{fulllineitems}

\index{updateSuperCompositeComponent() (Modeling.AbstractComponents.Component.Component method)}

\begin{fulllineitems}
\phantomsection\label{\detokenize{Modeling/Components:Modeling.AbstractComponents.Component.Component.updateSuperCompositeComponent}}\pysiglinewithargsret{\sphinxbfcode{updateSuperCompositeComponent}}{}{}
\end{fulllineitems}

\index{voltageOverMe() (Modeling.AbstractComponents.Component.Component method)}

\begin{fulllineitems}
\phantomsection\label{\detokenize{Modeling/Components:Modeling.AbstractComponents.Component.Component.voltageOverMe}}\pysiglinewithargsret{\sphinxbfcode{voltageOverMe}}{}{}
\end{fulllineitems}


\end{fulllineitems}



\chapter{Simulation}
\label{\detokenize{Simulation::doc}}\label{\detokenize{Simulation:simulation}}

\section{Simulation.NetToComp}
\label{\detokenize{Simulation:simulation-nettocomp}}\label{\detokenize{Simulation:module-Simulation.NetToComp}}\index{Simulation.NetToComp (module)}
Use the triangle class to represent triangles.
\index{NetToComp (class in Simulation.NetToComp)}

\begin{fulllineitems}
\phantomsection\label{\detokenize{Simulation:Simulation.NetToComp.NetToComp}}\pysiglinewithargsret{\sphinxstrong{class }\sphinxcode{Simulation.NetToComp.}\sphinxbfcode{NetToComp}}{\emph{filename}}{}
Bases: \sphinxcode{object}

Beispielkommentar
\index{getCommentsFromNetlist() (Simulation.NetToComp.NetToComp method)}

\begin{fulllineitems}
\phantomsection\label{\detokenize{Simulation:Simulation.NetToComp.NetToComp.getCommentsFromNetlist}}\pysiglinewithargsret{\sphinxbfcode{getCommentsFromNetlist}}{\emph{netListFile}}{}
Create a triangle with sides of lengths \sphinxtitleref{a}, \sphinxtitleref{b}, and \sphinxtitleref{c}.

Raises \sphinxtitleref{ValueError} if the three length values provided cannot
actually form a triangle.

\end{fulllineitems}

\index{getComponents() (Simulation.NetToComp.NetToComp method)}

\begin{fulllineitems}
\phantomsection\label{\detokenize{Simulation:Simulation.NetToComp.NetToComp.getComponents}}\pysiglinewithargsret{\sphinxbfcode{getComponents}}{}{}
Create a triangle with sides of lengths \sphinxtitleref{a}, \sphinxtitleref{b}, and \sphinxtitleref{c}.

Raises \sphinxtitleref{ValueError} if the three length values provided cannot
actually form a triangle.

\end{fulllineitems}

\index{parseCommentsToArgs() (Simulation.NetToComp.NetToComp method)}

\begin{fulllineitems}
\phantomsection\label{\detokenize{Simulation:Simulation.NetToComp.NetToComp.parseCommentsToArgs}}\pysiglinewithargsret{\sphinxbfcode{parseCommentsToArgs}}{\emph{args}, \emph{commentList}, \emph{name}}{}
\end{fulllineitems}

\index{stringArrToDict() (Simulation.NetToComp.NetToComp method)}

\begin{fulllineitems}
\phantomsection\label{\detokenize{Simulation:Simulation.NetToComp.NetToComp.stringArrToDict}}\pysiglinewithargsret{\sphinxbfcode{stringArrToDict}}{\emph{strArr}}{}
Create a triangle with sides of lengths \sphinxtitleref{a}, \sphinxtitleref{b}, and \sphinxtitleref{c}.

Raises \sphinxtitleref{ValueError} if the three length values provided cannot
actually form a triangle.

\end{fulllineitems}


\end{fulllineitems}



\chapter{Optimization}
\label{\detokenize{Optimization:optimization}}\label{\detokenize{Optimization::doc}}

\section{Optimizer}
\label{\detokenize{Optimization:optimizer}}\label{\detokenize{Optimization:module-Optimizations.OptimizationComponents.Optimizable}}\index{Optimizations.OptimizationComponents.Optimizable (module)}\index{Optimizable (class in Optimizations.OptimizationComponents.Optimizable)}

\begin{fulllineitems}
\phantomsection\label{\detokenize{Optimization:Optimizations.OptimizationComponents.Optimizable.Optimizable}}\pysiglinewithargsret{\sphinxstrong{class }\sphinxcode{Optimizations.OptimizationComponents.Optimizable.}\sphinxbfcode{Optimizable}}{\emph{comp\_names\_list}, \emph{paramname}, \emph{valfrom}, \emph{valto}, \emph{**kwargs}}{}
Bases: \sphinxcode{object}
\index{getOptimizableComponentNames() (Optimizations.OptimizationComponents.Optimizable.Optimizable method)}

\begin{fulllineitems}
\phantomsection\label{\detokenize{Optimization:Optimizations.OptimizationComponents.Optimizable.Optimizable.getOptimizableComponentNames}}\pysiglinewithargsret{\sphinxbfcode{getOptimizableComponentNames}}{}{}
\end{fulllineitems}

\index{getParamName() (Optimizations.OptimizationComponents.Optimizable.Optimizable method)}

\begin{fulllineitems}
\phantomsection\label{\detokenize{Optimization:Optimizations.OptimizationComponents.Optimizable.Optimizable.getParamName}}\pysiglinewithargsret{\sphinxbfcode{getParamName}}{}{}
\end{fulllineitems}

\index{getRangeBegin() (Optimizations.OptimizationComponents.Optimizable.Optimizable method)}

\begin{fulllineitems}
\phantomsection\label{\detokenize{Optimization:Optimizations.OptimizationComponents.Optimizable.Optimizable.getRangeBegin}}\pysiglinewithargsret{\sphinxbfcode{getRangeBegin}}{}{}
\end{fulllineitems}

\index{getRangeEnd() (Optimizations.OptimizationComponents.Optimizable.Optimizable method)}

\begin{fulllineitems}
\phantomsection\label{\detokenize{Optimization:Optimizations.OptimizationComponents.Optimizable.Optimizable.getRangeEnd}}\pysiglinewithargsret{\sphinxbfcode{getRangeEnd}}{}{}
\end{fulllineitems}

\index{getSetableList() (Optimizations.OptimizationComponents.Optimizable.Optimizable static method)}

\begin{fulllineitems}
\phantomsection\label{\detokenize{Optimization:Optimizations.OptimizationComponents.Optimizable.Optimizable.getSetableList}}\pysiglinewithargsret{\sphinxstrong{static }\sphinxbfcode{getSetableList}}{\emph{olist}}{}
\end{fulllineitems}

\index{getValue() (Optimizations.OptimizationComponents.Optimizable.Optimizable method)}

\begin{fulllineitems}
\phantomsection\label{\detokenize{Optimization:Optimizations.OptimizationComponents.Optimizable.Optimizable.getValue}}\pysiglinewithargsret{\sphinxbfcode{getValue}}{}{}
\end{fulllineitems}

\index{setValue() (Optimizations.OptimizationComponents.Optimizable.Optimizable method)}

\begin{fulllineitems}
\phantomsection\label{\detokenize{Optimization:Optimizations.OptimizationComponents.Optimizable.Optimizable.setValue}}\pysiglinewithargsret{\sphinxbfcode{setValue}}{\emph{v}}{}
\end{fulllineitems}

\index{toString() (Optimizations.OptimizationComponents.Optimizable.Optimizable method)}

\begin{fulllineitems}
\phantomsection\label{\detokenize{Optimization:Optimizations.OptimizationComponents.Optimizable.Optimizable.toString}}\pysiglinewithargsret{\sphinxbfcode{toString}}{}{}
\end{fulllineitems}


\end{fulllineitems}



\chapter{Indices and tables}
\label{\detokenize{index:indices-and-tables}}\begin{itemize}
\item {} 
\DUrole{xref,std,std-ref}{genindex}

\item {} 
\DUrole{xref,std,std-ref}{modindex}

\item {} 
\DUrole{xref,std,std-ref}{search}

\end{itemize}


\renewcommand{\indexname}{Python Module Index}
\begin{sphinxtheindex}
\def\bigletter#1{{\Large\sffamily#1}\nopagebreak\vspace{1mm}}
\bigletter{m}
\item {\sphinxstyleindexentry{Modeling.AbstractComponents.Component}}\sphinxstyleindexpageref{Modeling/Components:\detokenize{module-Modeling.AbstractComponents.Component}}
\item {\sphinxstyleindexentry{Modeling.CircuitSystemEquations}}\sphinxstyleindexpageref{Modeling/CircuitSystemEquations:\detokenize{module-Modeling.CircuitSystemEquations}}
\indexspace
\bigletter{o}
\item {\sphinxstyleindexentry{Optimizations.OptimizationComponents.Optimizable}}\sphinxstyleindexpageref{Optimization:\detokenize{module-Optimizations.OptimizationComponents.Optimizable}}
\indexspace
\bigletter{s}
\item {\sphinxstyleindexentry{Simulation.NetToComp}}\sphinxstyleindexpageref{Simulation:\detokenize{module-Simulation.NetToComp}}
\end{sphinxtheindex}

\renewcommand{\indexname}{Index}
\printindex
\end{document}